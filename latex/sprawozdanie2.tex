\documentclass[12pt]{article}

\usepackage[utf8]{inputenc}
\usepackage[T1]{fontenc}
\usepackage{listings}
\usepackage{amsmath, amssymb}
\usepackage{graphicx}
\usepackage{hyperref}
\usepackage{geometry}
\usepackage{caption}
\usepackage{tcolorbox}
\usepackage{graphicx}
\usepackage{booktabs}
\usepackage{float}     % for the H placement specifier

\DeclareCaptionLabelFormat{customTable}{Tabela #2}
\captionsetup[table]{
    labelformat=customTable,
    labelsep=colon
}

\DeclareCaptionLabelFormat{customFigure}{Wykres #2}
\captionsetup[figure]{
    labelformat=customFigure,
    labelsep=colon
}

\geometry{a4paper, margin=1in}

\lstset{
    inputencoding=utf8x,
    extendedchars=\true,
    literate={ą}{{\k{a}}}1
             {Ą}{{\k{A}}}1
             {ę}{{\k{e}}}1
             {Ę}{{\k{E}}}1
             {ó}{{\'o}}1
             {Ó}{{\'O}}1
             {ś}{{\'s}}1
             {Ś}{{\'S}}1
             {ł}{{\l{}}}1
             {Ł}{{\L{}}}1
             {ż}{{\.z}}1
             {Ż}{{\.Z}}1
             {ź}{{\'z}}1
             {Ź}{{\'Z}}1
             {ć}{{\'c}}1
             {Ć}{{\'C}}1
             {ń}{{\'n}}1
             {Ń}{{\'N}}1
}


\title{Struktury Baz Danych Projekt 2 - Sprawozdanie z indeksowej organizacji plików z użyciem struktury B-drzewa}
\author{Kamil Szabłowski, 193248}
\date{\today}

\begin{document}

\maketitle

\section{Wprowadzenie}
Zadaniem projektu była implementacja jednego z wybranych algorytmów organizacji plików indeksowych.
Zaimplementowanym przeze mnie algorytmem jest algorytm \emph{B-drzewa}.

Użytym w projekcie typem rekordu pliku są ciągi 10 \emph{liczb} z dodatkowycm kluczem, 
będącym liczbą natruralną większą od 0. Kryterium sortowania rekordu jest wartość klucza.
Za \emph{liczbę} będącą elementem rekordu uznaję 32-bitową liczbę całkowita ze znakiem. 
Poniżej przedstawione został przykład takiego rekordu:

\[
\begin{array}{|c|c|c|c|c|c|c|c|c|c|c|}
\hline
1 & 23 & -45 & 67 & 12 & -8 & 34 & 0 & 29 & -16 & 5 \\
\hline
\end{array}
\]

Pierwszy element powyższego rekordu to kluczz wynoszący 1, reszta stanowi ciąg 10 liczb.

Rozmiar rekordu wynosi w takim razie 44 bajtów (11 liczb całkowitych 4-bajtowych).
Przyjęty rozmiar bloku dyskowego pliku danych na potrzeby tego eksperymentu wynosi 364 bajtów.


\section{Opis implementacji}
Użytą w tym projekcie metodą organizacji pliku jest użycie struktury B-drzewa.
\\\\
Alogorytmy wykorzystane do wstawianie, usuwania i wyszukiwania rekordów w B-drzewie są analogiczne do tych
przedstawionych na wykładzie.
Modyfikacja rekordu polega na usunięciu starego rekordu i dodaniu nowego rekordu z nowym kluczem, chyba że
nie dokonujemy zmiany klucza, wtedy modyfikowana jest wartość rekordu bezpośrednio w pliku z danymi, po uwczesnym
wyszukaniu klucza.
\\\\
W mojej implementacji na stronę B-drzewa zapisaną do pliku składają się:
\begin{itemize}
    \item Wskaźnik na stronę rodzica (liczba całkowita, 4B)
    \item Liczba rekordów w stronie (liczba całkowita, 4B)
    \item Lewy skrajny wskaźnik na stronę, z rekordami mniejszymi od wszystkich innych należących do obecnej strony (liczba całkowita, 4B)
    \item Od \(1\) do \(2d\) węzłów, gdzie na dany węzeł składa się:
    \begin{itemize}
        \item Wartość klucza rekordu (liczba całkowita, 4B)
        \item Offset rekordu o danym kluczu w pliku z danymi (liczba całkowita, 4B)
        \item Wskaźnik na stronę z rekordami większymi od klucza w tym węźle, ale mniejszymi od klucza w następnym węźle (liczba całkowita, 4B)
    \end{itemize}
\end{itemize}

Nie zależnie od ilości węzłów w pliku indeksowym B-drzewa, zapisana (lub odczytana) strona
z dysku ma zawsze rozmiar \(4+4+4+2d*12 = 12+24d\) bajtów.

Strona w pamięci w programie jest prawie identyczna jak strona zapisana na dysku, z tą różnicą, że
lewy skrajny wskaźnik na stronę jest zastąpiony dodatkowym węzłem 0, w którym poza wartością wskaźnika na 
stronę-dziecko, wartość klucza i rekordu ustawiona jest na NULL

Jeśli chodzi o plik z danymi, to podobnie jak w przypadku pierwszego projektu, 
rekordy są zapisywane binarnie, jeden po drugim, bez żadnych dodatkowych informacji.

\subsection{Opis buforowania}
 
\section{Specyfikacja plików testowych}
Plikiem testowym jest plik tekstowy (.txt) o określonej poniżej strukturze.
\subsection{Struktura pliku testowego}
Plik testowy składa się z komend oddzielonych znakime nowej linii.
\\\\
Komendy dostępne do użycia w pliku testowym są analogiczne co do komend dostępnych w interfejsie tekstowym.
Lista dostępnych komend może zostać wyświetlona przy użyciu komendy \verb!help!. Dodatkowe komendy do debuggowania
programu są pokazane po użyciu komendy \verb!help debug!.
\\\\
Output komendy \verb!help!:

\begin{tcolorbox}[colframe=black!75, colback=white!95, title=Output komendy \texttt{help}]
\begin{verbatim}
List of available commands
TIP: Most commands can be used by using 
first letters of each word in the command
Example: 'am 5' is the same as 'addmulti 5'
--------------------------------------------------------------------
help                                     Show this help message
help debug                               Show help relating to debug
setcompensation [true/false]             Toggle compensation
clear                                    Clear all the files
rand [n]                                 Insert n random records
update [key] [value] [newKey: OPTIONAL]  Update record with given key
insert [key] [value]                     Insert record into the file
search [key]                             Search for record with key
print [group] [all]                      Prints all records in db
loadtest [filename]                      Loads test file
\end{verbatim}
\end{tcolorbox}

Output komendy \verb!help debug!:

\begin{tcolorbox}[colframe=black!75, colback=white!95, title=Output komendy \texttt{help debug}]
\begin{verbatim}
List of available debug commands
--------------------------------------------------------------------
dblockstats              Prints block stats
dforceflush              Forces a flush of the files
dgetrecord [n]           Gets a record at offset (data file)n
\end{verbatim}
\end{tcolorbox}


Przykład poprawnie ustrukturyzowanego pliku testowego:

\begin{tcolorbox}[colframe=black!75, colback=white!95, title=Plik: \texttt{text.txt}]
\begin{verbatim}
clear
insert 1 5
insert 2 10
insert 3 15
insert 4 20
insert 5 25
delete 2
search 3
search 2
dblockstats
print all group
print
\end{verbatim}
\end{tcolorbox}

Poniższy plik testowy wykona dokładnie to samo co powyższy:
\begin{tcolorbox}[colframe=black!75, colback=white!95, title=Plik: \texttt{text.txt}]
\begin{verbatim}
c
i 1 5
i 2 10
i 3 15
i 4 20
i 5 25
d 2
s 3
s 2
dbs
p a g
p
\end{verbatim}
\end{tcolorbox}

Tak spreparowany plik może bezproblemu zostać załadowany do interfejsu tekstowego.
\subsection{Załadowanie i uruchomienie pliku testowego}
Aby załadować plik testowy z dysku, wystarczy w interfejsie tekstowym użyć komendy \verb!loadtest nazwa_pliku.txt!,
gdzie zamiast \emph{nazwa\_pliku.txt} należy wpisać ścieżkę do docelowego pliku testowego, który ma zostać załadowany.
Ścieżka może być bezwzględna, lub względna - zależna od katalogu w którym został uruchomiony interfejs tekstowy.

\section{Sposób prezentacji wyników działania programu}

Po uruchomieniu programu, organizowany plik indeksowy orazz z danymi jest pusty i nie zawiera żadnych rekordów ani stron.

Aby dodać rekordy do bazy danych należzy użyć jednej z nastepujących komend:

\begin{itemize}
    \item \verb!insert [key] [value]! - próbuje dodać rekord o danym kluczu i wartości do bazy danych
    \item \verb!rand [n]! - próbuje dodać n losowych rekordów do bazy danych
\end{itemize}

Analogicznie, usunięcie rekordu z bazy danych odbywa się przy użyciu komendy \verb!delete [key]!.

Wyszukiwanie rekordu w bazie danych odbywa się przy użyciu komendy \verb!search [key]!.

Modyfikacja rekordu w bazie danych odbywa się przy użyciu komendy \verb!update [key] [value] [newKey: OPTIONAL]!

Po użyciu komendy \verb!print! bez żadnych parametrów, program wypisuje zawartość bazy danych w formie rekordów, od najmniejszego
do największego klucza w następującym formacie:
Przykład:
\begin{tcolorbox}[colframe=black!75, colback=white!95, title=Output komendy \texttt{print}]
\begin{verbatim}
(1) 10 10 10 10 10 10 10 10 10 10
(2) 20 20 20 20 20 20 20 20 20 20
(3) 30 30 30 30 30 30 30 30 30 30
(4) 40 40 40 40 40 40 40 40 40 40
(5) 50 50 50 50 50 50 50 50 50 50
\end{verbatim}
\end{tcolorbox}

Liczba w nawiasie to klucz rekordu, podczas gdy liczby po spacji to wartości rekordu, czyli liczby należące do ciągu.

Po użyciu komendy \verb!print! z parametrem \verb!group!, program wypisuje zawartość bazy danych w formie rekordów,
ale tym razem grupując rekordy znajdujące się na jednej stronie B-drzewa.

Po użyciu komendy \verb!print! z parametrem \verb!all!, program wypisuje zawartość bazy danych w pełnej formie. Przykład:
\begin{tcolorbox}[colframe=black!75, colback=white!95, title=Output komendy \texttt{print all}]
\begin{verbatim}
RecordOffset: 0 | PageOffset: 0 | ParentPageOffset: 1 
| LeftPagePtr: -1 | RightPagePtr: -1
(1) 1 1 1 1 1 1 1 1 1 1 

RecordOffset: 1 | PageOffset: 0 | ParentPageOffset: 1 
| LeftPagePtr: -1 | RightPagePtr: -1
(2) 2 2 2 2 2 2 2 2 2 2 

RecordOffset: 2 | PageOffset: 1 | ParentPageOffset: -1 
| LeftPagePtr: 0 | RightPagePtr: 2
(3) 3 3 3 3 3 3 3 3 3 3 

RecordOffset: 3 | PageOffset: 2 | ParentPageOffset: 1 
| LeftPagePtr: -1 | RightPagePtr: -1
(4) 4 4 4 4 4 4 4 4 4 4 

RecordOffset: 4 | PageOffset: 2 | ParentPageOffset: 1 
| LeftPagePtr: -1 | RightPagePtr: -1
(5) 5 5 5 5 5 5 5 5 5 5 

-----------------------------------
Total records: 5
Total pages count: 3
Height: 2
-----------------------------------
\end{verbatim}
\end{tcolorbox}

Poza wyświetleniem klucza i wartości rekordu, program wypisuje również informacje o stronach B-drzewa, 
takie jak:
\begin{itemize}
    \item \verb!RecordOffset! - offset rekordu w pliku z danymi
    \item \verb!PageOffset! - offset strony w pliku indeksowym
    \item \verb!ParentPageOffset! - offset strony rodzica w pliku indeksowym
    \item \verb!LeftPagePtr! - offset strony dziecka z rekordami mniejszymi od klucza w obecnym węźle
    \item \verb!RightPagePtr! - offset strony dziecka z rekordami większymi od klucza w obecnym węźle
\end{itemize}

Wartość \(-1\) w polach \verb!LeftPagePtr!, \verb!RightPagePtr! oraz \verb!ParentPageOffset! oznacza, 
że dany wskaźnik nie wskazuje na nic, czyli węzeł nie ma dzieci, bądź rodzica (tylko korzeń).

Dodatkowo wyświetlana jest informacja o całkowitej liczbie rekordów, 
liczbie stron oraz wysokości drzewa.


Po użyciu komendy \verb!print all group! program wypisuje zawartość bazy danych w pełnej formie, 
ale tym razem grupując rekordy po stronach B-drzewa. Przykład:
\begin{tcolorbox}[colframe=black!75, colback=white!95, title=Output komendy \texttt{print all group}]
\begin{verbatim}
-------- Page 0 ----------
RecordOffset: 0 | PageOffset: 0 | ParentPageOffset: 1 
| LeftPagePtr: -1 | RightPagePtr: -1
(1) 1 1 1 1 1 1 1 1 1 1 

RecordOffset: 1 | PageOffset: 0 | ParentPageOffset: 1 
| LeftPagePtr: -1 | RightPagePtr: -1
(2) 2 2 2 2 2 2 2 2 2 2 

-------- Page 1 ----------
RecordOffset: 2 | PageOffset: 1 | ParentPageOffset: -1 
| LeftPagePtr: 0 | RightPagePtr: 2
(3) 3 3 3 3 3 3 3 3 3 3 

-------- Page 2 ----------
RecordOffset: 3 | PageOffset: 2 | ParentPageOffset: 1 
| LeftPagePtr: -1 | RightPagePtr: -1
(4) 4 4 4 4 4 4 4 4 4 4 

RecordOffset: 4 | PageOffset: 2 | ParentPageOffset: 1 
| LeftPagePtr: -1 | RightPagePtr: -1
(5) 5 5 5 5 5 5 5 5 5 5 

-----------------------------------
Total records: 5
Total pages count: 3
Height: 2
-----------------------------------
\end{verbatim}
\end{tcolorbox}

Jak widać powyżej, rekordy są grupowane po stronach B-drzewa, co w przypadku większej ilości rekordów może spowodować,
że rekordy nie są wyświetlone w kolejności rosnącej klucza.

\section{Eksperyment}
\subsection{Konstrukcja eksperymentu}
Do przeprowadzenia eksperymentu wykorzystane zostały funkcje udostępniane prze interfejs
tekstowy programu.

Schemat eksperymentu wyglądał następująco:

\begin{enumerate}
    \item Wygenerowanie plików testowych z losowymi rekordami
    \item Załadowanie i posortowanie plików testowych
    \item Spisanie informacji o każdym sortowaniu do pliku tekstowego
    \item Wyznaczenie teoretycznych liczb faz i operacji dyskowych na podstawie spisanych informacji
    \item Wygenerowanie wykresów na podstawie danych eksperymentalnych i teoretycznych
    \item Porównanie otrzymanych wyników
\end{enumerate}

\subsection{Przygotowania do eksperymentu}
Aby wygenerować pliki testowe przy użyciu interfejsu tekstowego, należało użyć komend: 
\begin{verbatim}
    clear
    rand N
    save randN.txt
\end{verbatim}
gdzie N to liczba rekordów do wygenerowania. Na potrzeby tego eksperymentu zostało
wygenerowane 7 plików testowych dla \(N = \{2^4, 2^6, 2^8, 2^{10}, 2^{12}, 2^{14}, 2^{16}\}\).
Komenda \verb|rand| generuje N rekordów, w których każdy elemnt przyjmuje wartość z zakresu od 0 do 999. 
Do eksperymentu użyłem liczb rekordów, które pozwalają na łatwiejsze obliczenie teoretycznej liczby
operacji dyskowych.
\\ \\
Aby wczytać i posortować utworzone pliki, należało skorzystać z poniższych komend:
\begin{verbatim}
    clear
    load randN.txt
    sort polyphase quiet
\end{verbatim}
Opcja \verb|quiet| sprawia, że jedynie informacje o danej fazie są wypisywane w konsoli, gdyż
dla większych N wypisywanie zawartość taśm zajmowałoby zbyt długo.

Po każdym sortowaniu, spisałem do osobnego pliku dane potrzebne w dalszej części eksperymentu, czyli
liczbę odczytów i zapisów z dysku, liczbę faz, początkową liczbę runów w pliku.


\subsection{Wyniki eksperymentu}

\subsection{Podsumowanie}
Przeprowadzony eksperyment potwierdza poprawność implementacji alogrytmu sortowania polifazowego z wykorzystaniem 
liczb Fibonacciego. Potwierdza to fakt, że obliczone liczby teoretyczne zgadzają się z wynikami otrzymanymi podczas eksperymentu.
\\ \\
Dla dużych plików początkowych jedank, uwidocznił się pewnien błąd względny wynoszący ok. \(2.5\%\) przy liczbie operacji dyskowych.
Wskazuje on na możliwość wykonywania przez program zbędnych operacji, lecz analiza kodu programu nie uwidoczniła dlaczego
taki błąd zaistniał.

Liczba faz za to wydaje się być w pełni poprawna, gdyż błąd między wartościami teoretycznymi i praktycznymi jest w oczekiwanym zakresie.

\end{document}